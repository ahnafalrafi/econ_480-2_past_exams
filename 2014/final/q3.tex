\section{Weak Instruments}
Suppose that
\begin{align*}
  Y = & \ X \beta + U \\
  X = & \ Z \gamma + V
\end{align*}
where \(Y, X, Z, U, V\) are scalar random variables, \(\beta, \gamma\) are
unknown scalar parameters, \(Z\) is independent of \(U\) and \(V\), \(\mathbb{E}
[U] = \mathbb{E} [V] = 0\) and
\[
  \mathbb{E} \left[ U^{2} \right] = \sigma_{U}^{2}, \mathbb{E} \left[ V^{2}
  \right] = \sigma_{V}^{2}, \mathbb{E} [U V] = \sigma_{U V}, \mathbb{E} \left[
  X^{2} \right] = M_{X} < \infty, \mathbb{E} \left[Z^{2} \right] = M_{Z} <
  \infty
\]
Let \(\left\{ Y_{i}, X_{i}, Z_{i} \right\}_{i = 1}^{n}\) be an IID sample from
the distribution of \((Y, X, Z)\). The instrumental variables estimator of
\(\beta\) is
\[
  \widehat{\beta}_{n} = \frac{\sum_{i = 1}^{n} Y_{i} Z_{i}}{\sum_{i = 1}^{n}
  X_{i} Z_{i}}
\]
The instrument \(Z\) is called weak if \(\gamma\) is close to zero. This can be
formalized by setting \(\gamma = \frac{c}{\sqrt{n}}\), where \(c \neq 0\) is a
constant.
\begin{enumerate}
  \item Is \(\widehat{\beta}_{n}\) consistent for \(\beta\) if \(\gamma =
  \frac{c}{\sqrt{n}}\), for some \(c \neq 0\)?
  \item Find the value of \(\alpha\) such that \(n^{\alpha} \left(
    \widehat{\beta}_{n} -  \beta \right)\) has a non-degenerate limiting
  distribution as \(n \to \infty\).
  \item Is the limiting distribution normal? If yes, find the mean and variance
  of the limiting distribution. If no, you need not obtain an analytic
  expression for the limiting distribution, but you must explain the reason for
  your answer.
\end{enumerate}

\subsection{Consistency of the instrumental variables estimator with weak
instruments}

The IV estimator is inconsistent in this case. Using the model, we have
\[
  \widehat{\beta}_{n} = \frac{\sum_{i = 1}^{n} Y_{i} Z_{i}}{\sum_{i = 1}^{n}
  X_{i} Z_{i}} = \beta + \frac{\sum_{i = 1}^{n} U_{i} Z_{i}}{\frac{c}{\sqrt{n}}
  \sum_{i = 1}^{n} Z_{i}^{2} + \sum_{i = 1}^{n} V_{i} Z_{i}}
\]
Dividing both numerator and denominator by \(\sqrt{n}\), we have
\[
  \widehat{\beta}_{n} = \beta + \frac{\frac{1}{\sqrt{n}} \sum_{i = 1}^{n} U_{i}
  Z_{i}}{\frac{c}{n} \sum_{i = 1}^{n} Z_{i}^{2} + \frac{1}{\sqrt{n}} \sum_{i =
  1}^{n} V_{i} Z_{i}}
\]
We start with applying KSLLN2 to the first term in the denominator. Since
\(\mathbb{E} \left[ Z^{2} \right] = M_{Z} < \infty\), it follows that
\[
  \frac{c}{n} \sum_{i = 1}^{n} Z_{i}^{2} \overset{\mathrm{a.s.}}{\to} c
  \mathbb{E} \left[ Z^{2} \right] = c M_{Z}
\]
For the remaining terms, we will apply the Multivariate LLCLT. First, note that
since \(\mathbb{E} [U] = \mathbb{E} [V] = 0\), by independence of \(Z\) to \((U,
V)\), it follows that
\begin{align*}
  \mathbb{E} [U Z] = & \ \mathbb{E} [U] \mathbb{E} [Z] = 0 \cdot \mathbb{E} [Z]
  = 0 \\
  \mathbb{E} [V Z] = & \ \mathbb{E} [V] \mathbb{E} [Z] = 0 \cdot \mathbb{E} [Z]
  = 0 \\
  \mathrm{Var} [U Z] = \mathbb{E} \left[ U^{2} Z^{2} \right] = & \ \mathbb{E}
  \left[ U^{2} \right] \mathbb{E} \left[ Z^{2} \right] = \sigma_{U}^{2} M_{Z} <
  \infty \\
  \mathrm{Var} [V Z] = \mathbb{E} \left[ V^{2} Z^{2} \right] = & \ \mathbb{E}
  \left[ V^{2} \right] \mathbb{E} \left[ Z^{2} \right] = \sigma_{V}^{2} M_{Z} <
  \infty \\
  \mathrm{Cov} [U Z, V Z] = \mathbb{E} [(U Z) \cdot (V Z)] = \mathbb{E} \left[ U
  V Z^{2} \right] = & \ \mathbb{E} [U V] \cdot \mathbb{E} \left[ Z^{2} \right] =
  \sigma_{U V} M_{Z} \in \mathbb{R}
\end{align*}
By the Multivariate Lindeberg-L\'evy CLT,
\[
  \left( \begin{array}{c}
    \frac{1}{\sqrt{n}} \sum_{i = 1}^{n} U_{i} Z_{i} \\
    \frac{1}{\sqrt{n}} \sum_{i = 1}^{n} V_{i} Z_{i}
  \end{array} \right) = \frac{1}{\sqrt{n}} \sum_{i = 1}^{n}
  \left( \begin{array}{c}
    U_{i} Z_{i} \\
    V_{i} Z_{i}
  \end{array} \right) \overset{d}{\to} \mathcal{N} \left( \mathbf{0}_{2
  \times 1}, \Sigma \right)
\]
where
\[
  \Sigma = M_{Z} \cdot \left( \begin{array}{cc}
    \sigma_{U}^{2} & \sigma_{U V} \\
    \sigma_{U V} & \sigma_{V}^{2}
  \end{array} \right)
\]
By Slutsky's Theorem,
\begin{align*}
  \left( \begin{array}{c}
    \frac{1}{\sqrt{n}} \sum_{i = 1}^{n} U_{i} Z_{i} \\
    \frac{c}{n} \sum_{i = 1}^{n} Z_{i}^{2} + \frac{1}{\sqrt{n}} \sum_{i = 1}^{n}
    V_{i} Z_{i}
  \end{array} \right) = & \ \left( \begin{array}{c}
    0 \\
    \frac{c}{n} \sum_{i = 1}^{n} Z_{i}^{2}
  \end{array} \right) + \frac{1}{\sqrt{n}} \sum_{i = 1}^{n}
  \left( \begin{array}{c}
    U_{i} Z_{i} \\
    V_{i} Z_{i}
  \end{array} \right) \\
  \overset{d}{\to} & \ W = \left( \begin{array}{c}
    W_{1} \\
    W_{2}
  \end{array} \right) \sim \mathcal{N} \left( \left( \begin{array}{c}
      0 \\
      c \mathbb{E} \left[ Z^{2} \right]
  \end{array} \right), \Sigma \right)
\end{align*}
Define the transformation \(g \left( w_{1}, w_{2} \right) = \beta +
\frac{w_{1}}{w_{2}}\). Since \(\mathbb{P} \left( W_{2} = 0 \right) = 0\), by the
CMT
\[
  \widehat{\beta}_{n} \overset{d}{\to} g \left( W_{1}, W_{2} \right) = \beta +
  \frac{W_{1}}{W_{2}}
\]
Since the distribution limit is stochastic, \(\widehat{\beta}_{n}
\overset{p}{\not \to} \beta\).

\begin{remark*}
The convergence in distribution arguments can be reformulated in a number of
equivalent ways. For instance, we could say something along the lines of
\[
  \widehat{\beta}_{n} \overset{d}{\to} \beta + \frac{A_{1}}{c \mathbb{E} \left[
  Z^{2} \right] + A_{2}}
\]
where
\[
  \left( \begin{array}{c}
    A_{1} \\
    A_{2}
  \end{array} \right) \sim \mathcal{N} \left( \mathbf{0}_{2 \times 1}, \Sigma
  \right)
\]
This can be achieved by using a slightly different sequencing of our asymptotic
arguments. The conclusions though are effectively the same. Notice that though
there are different ways of getting to the same answer, there is only one
correct conclusion in the end.
\end{remark*}

\subsection{Rate of convergence to a non-degenerate limit distribution}

Clearly from the first part of the question, we have \(n^{\alpha} \left(
\widehat{\beta}_{n} - \beta \right) \overset{d}{\to} \beta +
\frac{W_{1}}{W_{2}}\) with \(\alpha = 0\).

\subsection{Non-normality of the limit distribution}

Again, the limit distribution is not normal, since it is \(\beta +
\frac{W_{1}}{W_{2}}\) where \(W = \left( W_{1}, W_{2} \right)^{\prime}\) follows
a bivariate normal distribution (and the ratio of the components of a bivariate
normal distribution is not normal).
